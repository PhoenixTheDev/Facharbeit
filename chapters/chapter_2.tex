\newacronym{ide}{IDE}{Intelligent Development Environment}

\chapter{Grundlagen}

Java ist eine objektorientierte, kompilierte Programmiersprache, welche plattformunabhängig ist. Sie wurde von der Programmiersprache C inspiriert, weshalb Java eine ähnliche Syntax hat. Java erschien 1995 und wird jeher für viele unterschiedliche Dinge genutzt. Auf der Internetseite von Oracle wird geschrieben: „Mit Java können Sie Online-Spiele spielen, mit Menschen auf der ganzen Welt chatten, Ihre Hypothekenzinsen berechnen und Bilder in 3D betrachten, um nur einige Beispiele zu nennen“\cite{Oracle:Java}. Dazu ist Java auf 13 Milliarden Geräten vertreten, und über 20 Jahre unter den drei meistgenutzten Programmiersprachen der Welt \cite{Github:PYPL}. \cite{Louis:2010}\cite{Oracle:Twitter} \par
Python wurde in den 90er Jahren von Guido van Rossum entwickelt. Er war an der Entwicklung von ABC beteiligt und hat mit der positiven und negativen Kritik Python entwickelt. Ursprünglich war sie als Skriptsprache für Amoeba gedacht. Der Name von für die Programmiersprache kommt von den britischen Komikern \textit{Monty Python}. Python ist, wie Java, ebenfalls eine objektorientierte, plattformunabhängige, interpretierte High-Level Programmiersprache. Wie ABC soll Python einfach zu verstehen und zu lesen sein. Bei Python wurde stark darauf geachtet, dass die Programmiersprache mächtig ist, was bei ABC nicht der Fall war und oft kritisiert wurde. Mit kleinen und übersichtlichen Programmen ist man in der Lage komplexe Aufgaben zu lösen. Python kann für die Web- und App-Entwicklung genutzt werden. Durch die Unmengen an Bibliotheken ist die Programmierung in Python bei neuen Themen wie künstlicher Intelligenz, Maschine Learning und Deep Learning besonders beliebt \cite{PythonCS}. Das liegt auch daran, dass Python Bibliotheken so einfach wie möglich bereitstellt. Die Standartbibliothek ist sehr umfangreich. Beispielsweise das Summieren jedes Elementes eines Arrays wird in Python mit einer einfachen Methode gemacht, während bei Java auf komplexere Wege zurückgegriffen werden muss. \cite{Python3:Buch}

\section{Grundbausteine und Funktionen}

Für den ausführlichen Vergleich der beiden Programmiersprachen, werden Syntax, Codelänge, Arbeitsspeicherauslastung und Laufzeit verglichen und daraus ein Resultat gezogen.
In Java wird der Einstiegspunkt durch das \textit{\acrlong{ide}} (IDE) BlueJ definiert. Der Startpunkt für das Programm wird von IDE zu IDE unterschiedlich definiert. Die IDE \textit{pyCharm}, welche die fünft meistgenutzte IDE der Welt ist, führt entweder eine \textit{main.py} Datei aus oder es wird manuell ein Einstiegspunkt definiert \cite{Github:IDE}. Das geschieht wie in \ref{lst:pystart} gezeigt.

\begin{lstlisting}[language=python,caption={Einstiegspunkt Python},captionpos=b,label={lst:pystart}]
if __name__ == '__main__':
\end{lstlisting}

Der Code wird eingerückt geschrieben. Der Einstiegspunkt ist nicht in einer Klasse enthalten. Wird das Programm bei einem manuellen Einstiegspunkt gestartet, wird alles Eingerückte ausgeführt, bis es unterbrochen wird. Wenn die \textit{main.py} durchlaufen wird, läuft das Dokument von oben bis unten durch. Klassen und Methoden werden beim Durchlaufen ignoriert. Dadurch wird keine Klasse gebraucht und kleine Programme können schnell programmiert werden. Eine Klasse ist ein Entwurf mit Eigenschaften und Funktionen \cite{gfg}. In Java ist eine Klasse zum Ausführen des Programms essenziell und deswegen werden als erstes die Unterschiede von Klassen gegenübergestellt.

\begin{lstlisting}[language=java,caption={Klasse in Java},captionpos=b,label={lst:java:class}]
public class Tisch
{
    public Tisch()
    {
        System.out.println("Tisch");
    }
}
\end{lstlisting}

\begin{lstlisting}[language=python,caption={Klasse in Python},captionpos=b,label={lst:python:class}]
class Tisch:
    def __init__(self):
        print('Tisch')
\end{lstlisting}

Lisiting \ref{lst:java:class} und Listing \ref{lst:python:class} schreiben, sobald sie initialisiert werden, „Tisch“ in die Konsole. Bei dieser direkten Gegenüberstellung der beiden Klassen fallen einige Unterschiede in der Syntax auf. Eine Klasse in Python wird nicht als \textit{public} oder \textit{private} definiert, sondern ist immer public. In der Klasse ist ein Konstruktor enthalten. Dieser wird unter den gleichen Bedingungen aufgerufen wie bei Java. Der Konstruktor besteht jeweils aus einem Kopf und einen Rumpf. Der Kopf in Java besteht aus dem Schlüsselbegriff \textit{public} oder \textit{private}, dem Namen der Klasse und Parameterklammern. Anders ist das in Python, wo der Konstruktor aus den Schlüsselbegriffen \textit{def} und \textit{\_\_init\_\_}, sowie den Parameterklammer und einem Doppelpunkt besteht. Der Code im Konstruktor wird beim Initialisieren der Klasse ausgeführt. Der Code endet in Java mit einem Semikolon und in Python nicht. Besonders durch die fehlenden geschweiften Klammern, sondern dem Definieren der Rümpfe durch das Einrücken des Codes ist Python um vier Zeilen kürzer. Sofern man sich an die Formatierungsregeln hält. Aber weniger Zeichen hat Python auch. Java braucht für den Code 60 Zeichen, während in Python nur 44 gebraucht werden. \par

Ein Tisch hat verschiedene Attribute. Darunter fallen Länge, Breite, Höhe, Anzahl der Beine und Dicke des Tisches. Diese Werte werden in der \textit{Random-Access Memory} (RAM) gespeichert \cite{Louis:2010}. Eine Variable hält entweder einen Wert (primitiv) oder eine Referenz auf den Wert (Referenz). Folglich ist eine Variable ein Speicherplatz. Der Wert im Speicher kann immer wieder geändert werden. In Java müssen die Werte immer mit dem Typ übereinstimmen. Java hat vier primitive Datentypen. \textit{Int, double, char und boolean} müssen immer einen Wert haben. \textit{String} hingegen greift über eine Referenz auf den Wert zu und kann deswegen auch \textit{Null} sein. Python hat keine Datentypen. Eine Variable kann immer jede Art von Wert speichern. Wie in jeder Sprache haben auch Programmiersprachen eine Art Rechtschreibung. Variablen werden nach bestimmten Richtlinien benannt, ein leichteres Lesen des Codes ermöglichen. Nach diesen \textit{naming conventions} werden Variablen in Java nach \textit{lowerCamelCase} benannt \cite{Microsoft:CapCon}. Auch die \textit{naming convention} ist in Python anders. Bei der Benennung von Variablen wird sich an \textit{snake\_case} gehalten \cite{Ims:h-s}. \cite{JavaNC}\cite{PythonStyle}\cite{JVMS}

\begin{lstlisting}[language=java,caption={Variablen in Java},captionpos=b,label={lst:java:variablen}]
public class Tisch
{
    int laenge;
    int breite = 50;
    int hoehe = 80;
    
    public Tisch()
    {
        laenge = 100;
        System.out.println("Tisch");
    }
}
\end{lstlisting}

\begin{lstlisting}[language=python,caption={Variablen in Python},captionpos=b,label={lst:python:variablen}]
class Tisch:
    def __init__(self):
        self.laenge = 100
        self.breite = 50
        self.hoehe = 80
        print('Tisch')
\end{lstlisting}

Durch die Gegenüberstellung beider Versionen mit Variablen fallen einige Dinge auf. Die globalen Variablen werden im Konstruktor initialisiert und deklariert. Das \textit{self.} macht die Variablen global. Ohne das \textit{self.} sind die Variablen nur im Konstruktor nutzbar. Variablen die nur im Konstruktor oder in Methoden nutzbar sind werden auch temporäre Variablen genannt, da der Wert nur temporär ist und nach dem Konstruktor oder der Methode verworfen wird. Soll die Variable in anderen Codeteilen aufgerufen werden, muss in Java nur den Variablenname geschrieben werden, bei Python muss das \textit{self.} davor. \par
Durch das Angeben eines Datentypens in Java, werden in Python weniger Zeichen für die Initialisierung und Deklarierung von Variablen gebraucht. Die Unabhängigkeit von Datentypen in Python ist ein großer Vorteil gegenüber Java.\par
Ein Sonderfall ist der Array oder auch die Liste. Während Java die beiden einzeln sieht, kombiniert Python die beiden. Ein Array ist eine Sammlung von Speichern für Werte mit einer konstanten Länge. Eine Liste ist eine Sammlung mit flexibler Länge, es können also Werte entfernt werden und hinzugefügt werden. Bei einem Array kann mit einem Index auf den Speicher zugegriffen werden. Eine Liste ist eine Verkettung von Werten und es kann immer nur auf das nächste Element zugegriffen werden. Wenn man auf das fünfte Element des Arrays zugreifen möchte geht das per genullten Index. Bei einer Liste wird mit dem ersten Element angefangen und es wird bis zum fünften Element auf den Nachbarn zugegriffen. Deshalb kann auch immer nur mit dem jetzigen Element in einer Liste gearbeitet werden. \cite{NRWListe}\cite{Louis:2010}\cite{Python3:Buch}

\begin{lstlisting}[language=java,caption={Array in Java},captionpos=b,label={lst:java:array}]
public class Tisch
{
    int breite;
    int laenge = 100;
  
    String[] gegenstaende = new String[3];
  
    public Tisch()
    {
        breite = 50;
    
        gegenstaende[0] = "Buch";
        gegenstaende[1] = "Stift";
        gegenstaende[2] = "Pflanze";
    }
}
\end{lstlisting}

In Java wird auf die einzelnen Speicher per Index zugegriffen, wie Listing \ref{lst:java:array} zeigt. Nachteil davon ist, dass nicht auf Index 3 zugegriffen werden kann, dieser existiert nämlich nicht. Der Array hat eine konstante Länge, welche sich nicht mehr nach der Deklarierung ändern lässt. Alle drei Elemente sind auf dem Tisch und die Anzahl der Elemente lässt sich nicht mehr verändern.

\begin{lstlisting}[language=java,caption={Liste in Java},captionpos=b,label={lst:java:liste}]
public class Tisch
{
    int breite;
    int laenge = 100;
  
    String[] gegenstaende = new String[3];
  
    public Tisch()
    {
        breite = 50;
    
        gegenstaende.append("Buch");
        gegenstaende.append("Stift");
        gegenstaende.append("Pflanze");
    }
}
\end{lstlisting}

Die Liste ermöglicht ein Hinzufügen von Elementen per \textit{append} und ein Löschen per \textit{remove}. Entfernt werden kann aber immer nur das aktuelle Element. Es lässt sich nicht per Index auf die Elemente zugreifen.

\begin{lstlisting}[language=python,caption={Liste in Python},captionpos=b,label={lst:python:liste}]
class Tisch:
  def __init__(self):
    self.laenge = 100
    self.breite = 50
    self.gegenstaende = []
    self.gegenstaende.add('Buch')
    self.gegenstaende[0] = 'Faus'
    self.gegenstaende.add('Stift')
    self.gegenstaende.add('Pflanze')
    self.gegenstaende.remove('Pflanze')
    self.gegenstaende.pop(1)
\end{lstlisting}

Python erlaubt es mit dem \textit{add()} Befehl neue Elemente der Liste hinzuzufügen. Löschen funktioniert genauso gut. Entweder per \textit{pop()}, wo in den Parameterklammern ein Basis Null Index mitgegeben wird oder mit der Funktion \textit{remove()}. \textit{remove()} arbeitet mit einem Wert anstelle eines Indexes, welcher gelöscht werden soll. Von einer Liste mit zweimal dem gleichen Wert wird das erste Element gelöscht. Das Python in der Lage ist per Index auf die Elemente zuzugreifen erleichtert das Schreiben des Codes.\par
Das Erstellen eines Arrays in Java und der Liste in Python auf diese Art und Weise kann je nach Situation umständlich sein. Deshalb bieten beide Programmiersprachen eine vereinfachte Deklaration.\newpage

\begin{lstlisting}[language=java,caption={Einfache Deklarierung eines Arrays in Java},captionpos=b,label={lst:java:ezarray}]
String[] gegenstaende = {
    "Buch", "Stift", "Pflanze"
};
\end{lstlisting}

\begin{lstlisting}[language=python,caption={Einfach Deklarierung eines Arrays in Python},captionpos=b,label={lst:python:ezarray}]
gegenstaende = [
    'Buch', 'Stift', 'Pflnaze'
]
\end{lstlisting}

Dazu kommen noch weitere Funktionen für Listen von Python, welche weder in Arrays noch in den Listen von Java enthalten sind. Beispielsweise kann man anstelle eines Indexes auch einen Bereich. Das macht Python noch kürzer und übersichtlicher\par
Methoden sind Abfolgen von Anweisungen, welche per Methodenname aufgerufen werden können. Durch das Auskapseln von Code in Methoden entstehen mehrere Vorteile. Der Code lässt sich leichter wiederverwenden, das Programm ist insgesamt strukturierter und der geschriebene Code ist verständlicher. \cite{Python3:Buch}\cite{Louis:2010}

\begin{lstlisting}[language=java,caption={Methoden in Java},captionpos=b,label={lst:java:methode}]
public void verstelleTisch(double wert)
{
    hoehe = hoehe + wert;
}
\end{lstlisting}

\begin{lstlisting}[language=python,caption={Methode in Python},captionpos=b,label={lst:python:methode}]
def verstelle_tisch(wert):
    hoehe = hoehe + wert
\end{lstlisting}

Methoden in Python haben keinen Zugriffstypen, genauso wie bei Variablen sind Methoden immer \textit{public}. Da Python keine Datentypen hat haben Methoden auch keine Rückgabewerte. Der Methodenkopf besteht nur aus dem Schlüsselbegriff \textit{def}, dem Namen und Parametern, welche von Klammern umrahmt werden. Der Rumpf der Methode wird anders als in Java nur durch das Einrücken des Codes und dem Doppelpunkt am Ende des Methodenkopfes definiert. Wird der Code nicht eingerückt geschrieben wird er nicht der Methode zugeordnet. Parameter funktionieren in Java und Python gleich. Sie werden ebenfalls nicht mit einem Datentyp angegeben. Einen Wert von einer Methode zurück zu geben ist ebenfalls möglich. Der Rückgabewert kann bei Python vom Datentyp variieren, da die Methode nicht auf einen Datentyp festgelegt ist. Java lässt unterschiedliche Rückgabewerte in der gleichen Methode nicht zu. Laut der \textit{naming convention} werden Methoden nach den gleichen Prinzipen wie Variablen genannt \cite{Ims:h-s}\cite{Microsoft:CapCon}. \cite{Python3:Buch}\cite{Louis:2010}\par
Methoden können in Python nicht überladen werden. Dafür gibt es Standartparameter. Im Methodenkopf wird dem Parameter mit dem Gleichheitszeichen ein Wert zugewiesen

\begin{lstlisting}[language=python,caption={Methode in Python mit Standartparameter},captionpos=b,label={lst:python:sparameter}]
def methode(parameter = 'standartparameter'):
    print(parameter)
\end{lstlisting}

Beim Aufrufen der in Listing \ref{lst:python:sparameter} dargestellten Methode ist ein Parameter optional. Der Wert von \textit{parameter} ist “standartparameter“ wenn kein Wert angegeben wird. Wenn ein Wert angegeben wird, wird der Wert überschrieben.
Mit der jetzigen Methode \textit{verstelleTisch} kann die Höhe beliebt sein. Das kann mit einer If-Abfrage verhindert werden. If-Abfragen sind in der Lage zwischen verschieden Fällen zu unterscheiden. In diesem spezifischen Beispiel gibt es zwei Fälle. Entweder geht die neue Höhe außerhalb des Toleranzbereichs oder sie bleibt drin. Die If-Abfrage wird in Java und Python gleich strukturiert, die Syntax unterscheidet sich jedoch. \cite{Python3:Buch}\cite{Louis:2010}

\begin{lstlisting}[language=java,caption={If-Abfrage in Java},captionpos=b,label={lst:java:if}]
if (hoehe + wert < maxHoehe && hoehe + wert > minHoehe)
{
    hoehe = hoehe + wert;
}
\end{lstlisting}

\begin{lstlisting}[language=python,caption={If-Abfrage in Python},captionpos=b,label={lst:python:if}]
if hoehe + wert < maxHoehe && hoehe + wert > minHoehe:
    hoehe = hoehe + wert;
\end{lstlisting}

Listing \ref{lst:python:if} zeigt, dass der Doppelpunkt am Ende des Kopfes, sowie das Einrücken den Rumpf definieren.
Trotz der Ähnlichen Operatoren unterscheiden sich Python und Java bei den essenziellen Operatoren. \textit{\&\&} und \textit{||} werden in Python zu \textit{and} und \textit{or}. Die Funktionsweise bleibt gleich. Ebenfalls wichtig ist, dass das \textit{!} zu einem \textit{not} wird. Mit \textit{not} lassen sich genauso wie in Java \textit{boolean}-Werte umdrehen. \textit{Is} wird genutzt, um eine Variable mit \textit{None} zu vergleichen. Kombiniert mit \textit{not}, gibt \textit{is not} den Wert gedreht zurück. \cite{Python3:Buch}\cite{Louis:2010}
Die dazugehörigen \textit{else if} und \textit{else} Zweige gibt es in Python ebenfalls. Lediglich wird \textit{else if} zu \textit{elif}. Durch die Operatoren wird deutlich das Python besser zu lesen ist. Die Längen der If-Abfragen unterscheiden sich unwichtig minimal.\par
Als letztes werden die Schleifen in Java und Python gegenübergestellt. Schleifen wiederholen Aktionen solange bis eine Bedingung erfüllt ist. Dazu wird entweder die \textit{while}-Schleife oder die \textit{for}-Schleife genutzt.


