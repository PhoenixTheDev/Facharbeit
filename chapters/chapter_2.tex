\chapter{Grundlagen}

Java ist eine objektorientierte, kompilierte Programmiersprache, welche plattformunabhängig ist [4]. Sie wurde von der Programmiersprache C inspiriert, weshalb Java eine ähnliche Syntax hat [4]. Java erschien 1995 und wird jeher für viele unterschiedliche Dinge genutzt. Auf der Internetseite von Oracle wird geschrieben: „Mit Java können Sie Online-Spiele spielen, mit Menschen auf der ganzen Welt chatten, Ihre Hypothekenzinsen berechnen und Bilder in 3D betrachten, um nur einige Beispiele zu nennen“[5]. Dazu ist Java auf 13 Milliarden Geräten vertreten, und über 20 Jahre unter den drei meistgenutzten Programmiersprachen der Welt [7]. [2] \par
Python wurde in den 90er Jahren von Guido van Rossum entwickelt. Er war an der Entwicklung von ABC beteiligt und hat mit der positiven und negativen Kritik Python entwickelt. Ursprünglich war sie als Skriptsprache für Amoeba gedacht. Der Name von für die Programmiersprache kommt von den britischen Komikern Monty Python. Python ist, wie Java, ebenfalls eine objektorientierte, plattformunabhängige, interpretierte High-Level Programmiersprache. Wie ABC soll Python einfach zu verstehen und zu lesen sein. Bei Python wurde stark darauf geachtet, dass die Programmiersprache mächtig ist, was bei ABC nicht der Fall war und oft kritisiert wurde. Mit kleinen und übersichtlichen Programmen ist man in der Lage komplexe Aufgaben zu lösen. Python kann für die Web- und App-Entwicklung genutzt werden. Durch die Unmengen an Bibliotheken ist die Programmierung in Python bei neuen Themen wie künstlicher Intelligenz, Maschine Learning und Deep Learning besonders beliebt [25]. Das liegt auch daran, dass Python Bibliotheken so einfach wie möglich bereitstellt. Die Standartbibliothek ist sehr umfangreich. Beispielsweise das Summieren jedes Elementes eines Arrays wird in Python mit einer einfachen Methode gemacht, während bei Java auf komplexere Wege zurückgegriffen werden muss. [8]

\section{Grundbausteine und Funktionen}

In Listing \ref{lst:java:if} sehen Sie if.

\begin{lstlisting}[language=java,caption={If-Verzweigung in Java},captionpos=b,label={lst:java:if}]
if (hoehe + wert < maxHoehe && hoehe + wert > minHoehe)
{
    hoehe = hoehe + wert;
}
\end{lstlisting}

\begin{lstlisting}[language=python,caption={If-Verzweigung in Python},captionpos=b,label={lst:python:if}]
if hoehe + wert < max_hoehe and hoehe + wert > min_hoehe:
    hoehe = hoehe + wert
\end{lstlisting}
