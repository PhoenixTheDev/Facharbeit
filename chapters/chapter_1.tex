\chapter{Einführung} % erster Überblick über Programmiersprachen insgesamt
Programmiersprachen sind aus der modernen Zeit des 21. Jahrhunderts kaum noch wegzudenken, weil sie für alle elektronischen Geräte, gebraucht werden. 
Mithilfe von Programmiersprachen werden dem Computer Algorithmen, welche Abfolgen von Befehlen und Rechnung sind, mitgeteilt. 
Ein Compiler übersetzt die Programmiersprache in Maschinencode, welcher von der \textit{Central Processing Unit} (CPU) verstanden wird. 
Das Übersetzen von Programmiersprachen in Maschinencode erleichtert die Arbeit für Programmierer ungemein, da Programmiersprachen intuitiver, leichter zu schreiben und für den Menschen verständlicher und naheliegender sind als Maschinencode. 
Grace Hopper hat den ersten Compiler 1952 entwickelt und gilt somit als eine Pionierin der Programmierung.
\cite{Louis:2010}

\par % Wieso Python und wieso Java
Über die Zeit haben einige Programmiersprachen besonders an Bedeutung gewonnen. 
Zwei Programmiersprachen, die viel Beliebtheit genießen, sind Java und Python. 
Java wird unter Anderem am Nepomucenum Coesfeld zum jetzigen Zeitpunkt im Informatik-Unterricht der Jahrgangsstufe Q1 unterrichtet. 
Die Popularität von Python steigt Jahr für Jahr, da Python gerne für künstliche Intelligenz und \textit{Deep Learning} genutzt wird, mit der steigenden Nachfrage nach künstlicher Intelligenz \cite{Github:PYPL}\cite{Gray:2017}. 
Python hat Java 2018 als beliebteste Programmiersprache abgelöst. 
Es ist folglich vorteilhaft ein Grundwissen von Prinzipien und Syntax von Python zu haben. 

\par % was wird mit Python gemacht und wieso nicht mit Java. Warum ist es wichtig Python zu können.
Python wird im Rahmen dieser Facharbeit für die Implementierung eines hexagonalen Schachs praktisch angewandt. 
Auf eine gegenüberstellende Implementierung des hexagonalen Schachs mit Java wird verzichtet, da das zu umfangreich für die Facharbeit wäre. 
Ein besseres Verständnis wird durch Vorkenntnisse in der Programmiersprache Java ermöglicht, ist jedoch nicht essenziell. 

\par % woher kommt die Idee des hexagonalen Schachs
Die Idee zum Implementieren eines hexagonalen Schachs, um die Programmiersprache Python in einem praktischen Beispiel zu evaluieren, rührt von dem Video \textit{Hexagons are the Bestagons} von GCP Grey \cite{Grey:Bestagons}. 
In diesem wird erst ausführlich erörtert, weshalb das Hexagon die beste Form ist und danach wird hexagonales Schach als Beispiel für Spiele, welche von einer hexagonalen Abwandlung profitieren würden. 
Ob die hexagonale Variante von Schach und welche Programmiersprache besser ist, wird am Ende der Facharbeit herausgestellt.
