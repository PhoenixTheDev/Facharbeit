\chapter{Einführung}
Programmiersprachen sind aus der modernen Zeit des 21. Jahrhunderts kaum noch wegzudenken, weil sie für alle elektronischen Geräte, welche Abfolgen befolgen müssen, gebraucht werden. Sie ermöglichen es Programmierern dem Computer Algorithmen, welche eine Abfolge von Befehlen sind, mitzuteilen. Ein Compiler übersetzt die Programmiersprache in Maschinencode, welche von der \textit{Central Processing Unit} (CPU) verstanden werden. Das Übersetzen von Programmiersprachen in Maschinencode erleichtert die Arbeit für Programmierer ungemein, dadurch dass Programmiersprachen intuitiver, leichter zu schreiben und verständlicher sind als Maschinencode. Grace Hopper hat den ersten Compiler 1952 entwickelt und gilt somit als eine Pionierin der Programmierung. \par Über die Zeit haben einige Programmiersprachen besonders an Bedeutung gewonnen. Zwei Programmiersprachen, die viel Beliebtheit genießen, sind Java und Python. Java wird unter Anderem am Nepomucenum Coesfeld zum jetzigen Zeitpunkt im Informatik-Unterricht der Jahrgangsstufe Q1 unterrichtet. Die Popularität von Python steigt Jahr für Jahr, da Python gerne für künstliche Intelligenz und \textit{Deep Learning} genutzt wird, mit der steigenden Nachfrage nach künstlicher Intelligenz \cite{Github:PYPL}\cite{Gray:2017}. Python hat Java 2018 als beliebteste Programmiersprache abgelöst. Es ist folglich vorteilhaft ein Grundwissen von Prinzipien und Syntax von Python zu haben. \par Python wird im Rahmen dieser Facharbeit für die Implementierung eines hexagonalen Schachs praktisch angewandt. Auf eine gegenüberstellende Implementierung des hexagonalen Schachs mit Java wird verzichtet, da das den Rahmen sprengen würde. Ein besseres Verständnis wird durch Vorkenntnisse in der Programmiersprache Java ermöglicht, ist jedoch nicht essenziell. \par Die Idee zum Implementieren eines hexagonalen Schachs, um die Programmiersprache Python in einem praktischen Beispiel zu evaluieren, rührt von dem Video \textit{Hexagons are the Bestagons} von GCP Grey \cite{Grey:Bestagons}. In diesem wird erst ausführlich erörtert, weshalb das Hexagon die beste Form ist und danach werden hexagonale Spielabwandlungen aufgelistet. Darunter befindet sich auch Schach. Ob die hexagonale Variante von Schach besser ist und welche Programmiersprache gelernt werden sollte, wird am Ende der Facharbeit herausgestellt. \cite{Louis:2010}
