\chapter{Einführung}
Programmiersprachen sind in dem 21sten Jahrhundert unvermeidbar. Sie ermöglichen es Programmieren dem Computer Algorithmen, welche eine Abfolge von Befehlen sind, zu erteilen. In Raumschiffen, Autos, Computer oder Handys, aber auch in unscheinbareren Dingen wie einer Waschmaschine sind Programmiersprachen nicht mehr wegzudenken. Ein Compiler übersetzt die Programmiersprache in 1 und 0, welche von der Central Processing Unit (CPU) verstanden werden. Das Übersetzen von Programmiersprachen in 1 und 0 erleichtert die Arbeit für Programmierer ungemein. Grace Hopper hat den ersten Compiler 1952 entwickelt und somit gilt sie als eine Pionierin der Programmierung. Über die Zeit haben sich einige Programmiersprachen als besonders beliebt herausgefiltert. Zwei Programmiersprachen die viel Beliebtheit genießen sind Java und Python. Java wird unter anderem am Nepomucenum Coesfeld zum jetzigen Zeitpunkt in Informatik der Q1 unterrichtet. Das Ansehen von Python steigt Jahr für Jahr, durch die ebenfalls schnell ansteigende Nachfrage nach künstlicher Intelligenz \cite{Gray:2017}. Python hat Java 2018 als beliebteste Programmiersprache abgelöst. Es ist folglich vorteilhaft ein Grundwissen von Prinzipien und Syntax von Python zu haben. Python wird noch einmal für die Implementierung eines hexagonalen Schachs praktisch angewandt. Für Java wird ein generelles Grundverständnis vorausgesetzt, auf eine gegenüberstellende Implementierung mit Java wird deswegen verzichtet. Die Idee für das implementierte hexagonale Schach, um Python in Anwendung zu sehen, kommt durch das Video Hexagons are the Bestagons von GCP Grey \cite{Grey:Bestagons}. In diesem wird erst ausführlich erörtert, weshalb das Hexagon die beste Form ist und danach werden hexagonale Spielabwandlungen aufgelistet. Darunter befindet sich auch Schach. Gerade Schach ist besonders interessant, da durch die vielen verschiedenen Figuren und Züge spannende Situationen zustande kommen. Eine Abwandlung mit mehr Seiten und daraus resultierend mehr möglichen Zügen intensiviert das Spiel nochmal. Passend dazu wird das normale Schach mit der abgewandelten Version verglichen. \cite{Github:PYPL}\cite{Louis:2010}
