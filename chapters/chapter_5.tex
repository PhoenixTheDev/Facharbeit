\chapter{Zusammenfassung}

Python ist eine langsame Programmiersprache, weil sie interpretiert wird und Java kompiliert. Programme, die nicht auf Laufzeit optimiert werden müssen, sollten also in Python geschrieben werden. Python ist wegen der besonderen Syntax intuitiver. Durch wenig Code lässt sich schon schnell ein Programm programmieren. Die Bibliotheken helfen umso mehr. Ein Programm, welches viele Prozesse schnell ausführen soll, sollte auf jeden Fall in Java geschrieben werden. Bis ein Resultat erzielt wird dauert es Vergleichsweise länger, lohnt sich jedoch. Java und Python funktionieren auf jeder Plattform, was die Wahl einer Programmiersprache nicht auf Plattformen beschränkt.\par
Welche Programmiersprache man lernen sollte hängt von zwei Faktoren ab. Wenn schnell Erfolge erzielt werden sollen, ist Python zu empfehlen. Ist der Hintergedanke ein tiefgründiges Verständnis von Programmieren ist Java eher die Wahl. Optimal ist ein Mix aus den beiden Programmiersprachen. Mit Python sollte angefangen werden, um ein generelles Grundverständnis zu erlangen und Motivation für das Programmieren und die Informatik zu gewinnen. Danach Java für einen tieferen Einblick in die Prinzipien und Prozesse\par
Diese Aufteilung lässt sich auch auf die Schule anwenden. Schüler sind schneller motiviert, wenn sie schnell Ergebnisse erzielen und nicht erst das Grundgerüst aufbauen müssen. Von den Klassen sieben bis neun sollte Python gelehrt werden. Für die Oberstufe wird dann auf Java oder eine ähnliche Programmiersprache gewechselt. Wenn die Schüler in der zehnten Klasse einen Einblick auf die Q1 und Q2 bekommen, können sie sich dann festlegen, ob sie Informatik weiterwählen oder bei einem Grundverständnis bleiben.\par
Bei einem Vergleich der beiden Schachvarianten ist das hexagonale Schach meiner Meinung nach besser, dank der größeren Anzahl an möglichen Zügen. Es werden viele neue Möglichkeiten durch die zwei weiteren Seiten eröffnet. Hier empfehle ich nicht mit hexagonalem Schach in das Thema Schach einzusteigen. Besonders, weil das Vorhersagen und Planen von Zügen schwieriger ist. Ein erfahrener Spieler sollte sich die hexagonale Variante aber nicht entgehen lassen.
